\chapter{总结与展望}

\section{工作总结}
在这篇论文中,我提出了两个新的基于多知识库的表格实体链接方法。
两个方法的核心都是基于图的随机游走算法。
方法一的第一步是用一个基于图的迭代概率传播算法来进行单知识库实体链接。
在第二步中我提出了三条启发式规则来利用不同知识库实体间的``sameAs''关系来提升第一步的链接结果,同时也解决了多知识库实体链接结果冲突的问题。
方法二中使用了一个统一的图模型,直接将多知识库的实体和实体间的``sameAs''关系融合进实体消岐图,一步到位地计算出实体链接结果。
两个方法都有各自的优缺点。
优点在于二者都不依赖特定的信息 (表格的列头,知识库中的实体类型),
并且都是基于多知识库进行实体链接,弥补了单知识库实体覆盖程度不够的缺点。
方法一的缺点在于其第二步中的启发式规则的不稳定性。
方法二的缺点在于,在构建实体消岐图的时候,很多正确的参考实体由于``sameAs''关系缺失的原因不能和其他等价的实体进入同一个实体组结点,又因为链接的目标是为一个给定的指称选择一个实体组结点作为链接结果,作为这样就导致了最终链接结果会有所遗漏。
我设计并实现了两个方法与 TabEL\cite{bhagavatula2015tabel},LIEGE\cite{shen2012liege} 以及另外两个退化版本的方法的对比实验。
实验结果表明系统中的两个方法在不同的评价标准 (准确率、召回率、F1值和 MRR) 上表现得都非常优秀。
系统中的多知识库表格实体链接方法非常有效并且相对于单知识库实体链接有一个更好的实体覆盖度。
值得一提的是,系统中的两个方法都是通用的,可以使用任何单知识库或者相互链接的多知识库进行 Web 表格上的实体链接。


\section{未来展望}
在~\ref{challenge} 节中提到当前实体链接的关键挑战在于缺少基准数据集。
因此对于未来的工作,首要任务是建立更多的其他语言的基准数据集,用于开展基于多知识库的 Web 表格实体链接的新任务。
其次是改进优化系统中的方法,比如在计算实体链接影响因子的时候设计更多的特征,原先使用的特征主要反映的是指称与实体之间、实体与实体之间的语义相关度,更多有效特征的加入可以多维度地反映指称与实体之间、实体与实体之间的关系,从而提升链接的质量。
更进一步地,衡量本文中的方法在其他语言上,特别是英语上的效果。
除此之外,我还考虑给系统加上关系抽取的功能。
就像~\ref{task} 节中提到的,关系抽取是表格语义解释的主要任务之一。
根据表格中不同列实体链接的结果,通过两列某一行的关系即可得到表格列之间的关系,最终的结果表示为 RDF 三元组。
然后,我计划把系统中的方法进行封装,做成编程接口,提供 API 或者工具供他人使用,或者将这些接口做成 Web 应用,让他人通过 Web 就能轻松使用我的系统进行实体链接。
最后,我考虑将系统拓展成跨语言\cite{zhang2013cross} (Cross-lingual) 的多知识库表格实体链接系统。







